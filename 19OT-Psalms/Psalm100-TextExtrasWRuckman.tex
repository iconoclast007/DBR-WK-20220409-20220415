\chapter{Psalm 100}
\footnote{\textcolor[rgb]{0.00,0.25,0.00}{\hyperlink{TOC}{Return to end of Table of Contents.}}}\textcolor[rgb]{0.00,0.00,1.00}{A Psalm of praise.}\\
\\
\textcolor[rgb]{0.00,0.00,1.00}{Make a joyful noise unto the LORD, all ye lands.}\footnote{[RUCKMAN] We see the Millennium again. The “we” of
verse 3 are the Israelites, who at this point recognize the fulfillment of Ezekiel 34. Israel was literally a nation created by God (see Isa. 43:1). Of course we can make spiritual application. We can say that the New Testament believer is one of the Lord’s sheep (as in John 10:1–26 and 1 Pet. 2:25), and he is led by the “Good Shepherd” (see Ps. 23 and comments), but here the context is “come before his presence...into his courts” (vss. 2, 4). These are the familiar “courts” of Psalms 65:4; 84:2, 10; and 92:13.
Devotionally, three things are said about the Lord  \cite{Ruckman1992Psalms} : \begin{compactenum}
\item He is the One who created us (vs. 3, so
Mother Nature and “pools of proteins and
amino acids” had nothing to do with it).
\item He is good (vs. 5) despite His dealings,
which, at times, seem to come from a
malevolent despot (no irreverence: look at Job
9:24).
\item “His mercy is everlasting” (vs. 5), in the
sense that those to whom it is promised will
receive it eventually.
\item His truth (see Ps. 12:6–8 and 138:2)
“endureth to all generations” (vs. 5)
because it is “true from the beginning” (Ps.
119:160) and “settled in heaven” forever
(Ps. 119:89).
\end{compactenum}}

[3] \textcolor[rgb]{0.00,0.00,1.00}{Know ye that the LORD he \emph{is} God: \emph{it} \emph{is} he \emph{that} hath made us, and not we ourselves; \emph{we} \emph{are} his people, and the sheep of his pasture.}\footnote{[RUCKMAN] If “we are his people, and the sheep of his
pasture” (vs. 3), we can be thankful for the
ability to do all these things. There are
actually seven items in the list, but the last is
often overlooked because it is not connected
with “worship services.” The seventh item is
in verse 3. There, we are admonished to
know something; we are to “know ye that
the Lord...is he that made us.” That is a
great truth confirmed by Isaiah 45:12 and
Psalm 74:17. Perhaps your
great great grandfather was fired by the pack
for taking too many banana breaks, but ours
wasn’t. (Two monkeys were launched in a
rocket. After a while one of them said, “Well,
anyway, it beats the cancer clinic.”)
Some of you ought to buy two tickets when
you go to the zoo alone: one to get in and one
to get out. You may be a bankrupt monkey,
but I am not; and there are no tarsiers,
lemurs, or orangutans in my family line. I am
a “child of the King.” I am one of the Good
Shepherd’s sheep. I don’t believe in
evolution, but I must confess that I have been
tempted to believe it at times when I have
seen photos of Mike Tyson or Louis
Armstrong. (No offence! I didn't judge them
by their color; I judged them by their
photographs. Strictly individual judgment, on
an individual basis, man.) \cite{Ruckman1992Psalms}}
